\documentclass{article}
%% packages
%%i%%%%%%%%%%%%%%%%%%%%%%%%%%%%%%%%%%%%%%%%%%%%%%%%%%%%%%%%%%%%%%%%%%%%%%%%%%%%%%%
\usepackage{amssymb}
\usepackage{amsmath}
\usepackage{amsthm}%% proof etc
\usepackage{algorithm}
\usepackage[noend]{algorithmic}


\usepackage{hyperref} %% inclusion after algorithm: mandatory
\usepackage{url}

\usepackage{boxedminipage}
\usepackage{graphicx}%%[draft] : do not embed figs/picts
\usepackage{float}
\usepackage{rotating}

\usepackage{xspace}
\usepackage{nicefrac}
%\usepackage{underscore}

\usepackage{comment}
\usepackage{soul}
\usepackage{verbatim}%%block comment

\usepackage{csquotes} % \textquote{}

%% writing in French!
\usepackage[T1]{fontenc}    % for accents
\usepackage[utf8]{inputenc} % for coding
%\usepackage[french]{babel} % cesures, spaces etc
%% Misc 
\usepackage{xparse}   % \NewDocumentCommand
\usepackage{pifont}
\usepackage{marvosym}
%\usepackage{fp}       % floating point calculations: cf macro GanttLine
%\usepackage{eurosym}  % \euro symbol
%\usepackage{bibtopic}

\usepackage[usenames,dvipsnames]{xcolor}
\usepackage{listings}


\usepackage{fullpage}

%% macros
%%i%%%%%%%%%%%%%%%%%%%%%%%%%%%%%%%%%%%%%%%%%%%%%%%%%%%%%%%%%%%%%%%%%%%%%%%%%%%%%%%
\newif\ifSLIDES
\SLIDESfalse

\input{.latex_location_macros.sty}

%% General macros 
\input{\wmysty/macros-comments.sty}
\input{\wmysty/macros-envs.sty}
\input{\wmysty/macros-math-letters-fontified.sty}
\input{\wmysty/macros-symbols.sty}
\input{\wmysty/macros-algorithms-code.sty}

%% Macros specific to one broad topic/theme 
\input{\wmysty/macros-SBL.sty}
\input{\wmysty/macros-wp-delaunay-voronoi.sty}
\input{\wmysty/macros-wp-space-filling-models.sty}
\input{\wmysty/macros-wp-conformational-analysis.sty}

%% Macros specific to one/a few papers (PhD project)
%\renewcommand{\tored}{\color{black}}
%\renewcommand{\toblue}{\color{black}}

\title{TP}
\author{Jules}

\begin{document}
\maketitle

%=============================================


\section{Exo 1}

(a)

\texttt{x} est un int, donc la division \texttt{a/b} retourne un int.

On a donc \texttt{x : 3}

(b)

\texttt{d} est un float et \texttt{c} est un float donc \texttt{d = a/c}
retourne un float, on a \texttt{d : 0.75}

\texttt{e} est un float mais \texttt{a} et \texttt{b} sont des entiers donc 
\texttt{a/b} retourne un entier qui est ensuite casté en . On a \texttt{e : 0.00000}

\texttt{f} est un entier donc la division est entière. On a \texttt{f : 0}

\section{exo 2}

\begin{enumerate}
    \item \texttt{(5*x)+2*((3*b)+4)} $\rightarrow$ \texttt{98}
    \item \texttt{5*(x+2)*3*(b+4)} $\rightarrow$ \texttt{1850}
    \item \texttt{a==(b=6)} $\rightarrow$ \texttt{0}
    \item \texttt{a\%=d++} $\rightarrow$ \texttt{0}
    \item \texttt{a\%=++d} $\rightarrow$ \texttt{3}
    \item \texttt{a\&\&b || !0\&\&c \&\& !d} $\rightarrow$ \texttt{1}
    \item \texttt{((a\&\&b)||(!0\&\&c))\&\&!d} $\rightarrow$ \texttt{0}
    \item \texttt{((a\&\&b)||!0)\&\&(c\&\&!d)} $\rightarrow$ \texttt{0}
\end{enumerate}

\section{Exo3}

\begin{lstlisting}

    (3<=x && x<=6) || (7<=x && x<10)

\end{lstlisting}

\section{Exo 4}

\subsection{1}

\begin{lstlisting}
    if (a>b){
        if (a>10){
            printf("premier choix\n");
        }
        else if (b<10){
            printf("deuxieme choix\n");
        }
        else if (a==b){
            printf("troisieme choix\n");
        }
        else {
            printf("quatrieme choix\n");
        }
    }
\end{lstlisting}

\subsection{2}

\begin{itemize}
    \item premier choix : $(a>b)\wedge(a>10)$
    \item deuxième choix : $(a>v)\wedge (b<10)$
    \item troisième choix : $(a>b) \wedge (a=b) \equiv \bot $
    \item quatrième choix : $(a>b) \wedge (a<=10) \wedge (b>=10) \wedge (b\neq a) \equiv \bot$
\end{itemize}

\subsection{3}

On n'obtient pas de réponse pour $a<=b$

\section{exo 5}

\begin{enumerate}
    \item il manque un \texttt{;} à la fin de la ligne 2
    \item il manque les \texttt{break;} à chaque bloc du switch 
    et les expressions dans les \texttt{case} doivent être des constantes
    \item il manque les \texttt{break;} à chaque bloc du switch
\end{enumerate}

\section{Exo 6}



\end{document}