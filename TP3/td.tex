\documentclass{article}
%% packages
%%i%%%%%%%%%%%%%%%%%%%%%%%%%%%%%%%%%%%%%%%%%%%%%%%%%%%%%%%%%%%%%%%%%%%%%%%%%%%%%%%
\usepackage{amssymb}
\usepackage{amsmath}
\usepackage{amsthm}%% proof etc
\usepackage{algorithm}
\usepackage[noend]{algorithmic}


\usepackage{hyperref} %% inclusion after algorithm: mandatory
\usepackage{url}

\usepackage{boxedminipage}
\usepackage{graphicx}%%[draft] : do not embed figs/picts
\usepackage{float}
\usepackage{rotating}

\usepackage{xspace}
\usepackage{nicefrac}
%\usepackage{underscore}

\usepackage{comment}
\usepackage{soul}
\usepackage{verbatim}%%block comment

\usepackage{csquotes} % \textquote{}

%% writing in French!
\usepackage[T1]{fontenc}    % for accents
\usepackage[utf8]{inputenc} % for coding
%\usepackage[french]{babel} % cesures, spaces etc
%% Misc 
\usepackage{xparse}   % \NewDocumentCommand
\usepackage{pifont}
\usepackage{marvosym}
%\usepackage{fp}       % floating point calculations: cf macro GanttLine
%\usepackage{eurosym}  % \euro symbol
%\usepackage{bibtopic}

\usepackage[usenames,dvipsnames]{xcolor}
\usepackage{listings}

\lstdefinestyle{mystyle}{
    breakatwhitespace=false,         
    breaklines=true,                 
    captionpos=b,                    
    keepspaces=true,                 
    numbers=left,                    
    numbersep=5pt,                  
    showspaces=false,                
    showstringspaces=false,
    showtabs=false,                  
    tabsize=2,
    frame = single
}

\lstset{style=mystyle}


\usepackage{fullpage}

%% macros
%%i%%%%%%%%%%%%%%%%%%%%%%%%%%%%%%%%%%%%%%%%%%%%%%%%%%%%%%%%%%%%%%%%%%%%%%%%%%%%%%%
\newif\ifSLIDES
\SLIDESfalse

\input{.latex_location_macros.sty}

%% General macros 
\input{\wmysty/macros-comments.sty}
\input{\wmysty/macros-envs.sty}
\input{\wmysty/macros-math-letters-fontified.sty}
\input{\wmysty/macros-symbols.sty}
\input{\wmysty/macros-algorithms-code.sty}

%% Macros specific to one broad topic/theme 
\input{\wmysty/macros-SBL.sty}
\input{\wmysty/macros-wp-delaunay-voronoi.sty}
\input{\wmysty/macros-wp-space-filling-models.sty}
\input{\wmysty/macros-wp-conformational-analysis.sty}

%% Macros specific to one/a few papers (PhD project)
%\renewcommand{\tored}{\color{black}}
%\renewcommand{\toblue}{\color{black}}

\title{TP}
\author{Jules}

\newcommand{\graphscale}{.8}


\begin{document}
\maketitle

%=============================================
\section*{Exo 1}

\subsection*{(a)}

\begin{lstlisting}[caption=Séquence d'instruction (a)]
int a = 5; //le pointeur p n'est pas definit, c'est un pointeur null
*p = a;
\end{lstlisting}

\begin{figure}[H]
    \begin{center}
        \scalebox{\graphscale}{\includegraphics*[]{schem_mem_1-a.pdf}}
    \end{center}
    \caption{Schéma mémoire de la séquence d'instruction (a)}
\end{figure}

\begin{lstlisting}[caption=Séquence d'instruction (a) corrigée]
int *p, a = 5;
p = &a;
\end{lstlisting}

\begin{figure}[H]
    \begin{center}
        \scalebox{\graphscale}{\includegraphics*[]{schem_mem_1-a-2.pdf}}
    \end{center}
    \caption{Schéma mémoire de la séquence (a) corrigée}
\end{figure}

\subsection*{(b)}

\begin{lstlisting}[caption=Séquence d'instruction (b)]
int a=1;
int b=5;
int *p1,*p2;
float *p3;

p1=p2; // p2 n'a pas ete initialisee
p2=&a;
*p1=a; // on ne sais pas ou pointe p1
p3=&b; // p3 est un pointeur de float mais b est un int
(*p3)++;
\end{lstlisting}

\begin{lstlisting}[caption=Séquence d'instruction (b) corrigée]
int a = 1;
float b = 5.;
int *p1,*p2,*p3;

p2 = &a;
p1 = &a;

p3 = &b;
(*p3)++;
\end{lstlisting}

\begin{figure}[H]
    \begin{center}
        \scalebox{\graphscale}{\includegraphics*[]{schem_mem_1-b.pdf}}
    \end{center}
    \caption{Schéma mémoire de la séquence (b) corrigée}
\end{figure}


\subsection*{(c)}

\begin{figure}[H]
    \begin{center}
        \scalebox{\graphscale}{\includegraphics*[]{schem_mem_1-c.pdf}}
    \end{center}
    \caption{Schéma mémoire de la séquence d'instruction (c)}
\end{figure}

\section*{Exo 2}
\begin{figure}[H]
    \begin{center}
        \scalebox{\graphscale}{\includegraphics*[]{schem_mem_2.pdf}}
    \end{center}
    \caption{Schéma graphique de la séquence d'instruction de l'exercice 2}
\end{figure}

\section*{Exo machine 1}

\lstinputlisting{tp3-1.c}

\section*{Exo machine 2}

\lstinputlisting{tp3-2.c}

\end{document}