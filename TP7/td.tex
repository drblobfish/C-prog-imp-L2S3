\documentclass{article}
%% packages
%%i%%%%%%%%%%%%%%%%%%%%%%%%%%%%%%%%%%%%%%%%%%%%%%%%%%%%%%%%%%%%%%%%%%%%%%%%%%%%%%%
\usepackage{amssymb}
\usepackage{amsmath}
\usepackage{amsthm}%% proof etc
\usepackage{algorithm}
\usepackage[noend]{algorithmic}


\usepackage{hyperref} %% inclusion after algorithm: mandatory
\usepackage{url}

\usepackage{boxedminipage}
\usepackage{graphicx}%%[draft] : do not embed figs/picts
\usepackage{float}
\usepackage{rotating}

\usepackage{xspace}
\usepackage{nicefrac}
%\usepackage{underscore}

\usepackage{comment}
\usepackage{soul}
\usepackage{verbatim}%%block comment

\usepackage{csquotes} % \textquote{}

%% writing in French!
\usepackage[T1]{fontenc}    % for accents
\usepackage[utf8]{inputenc} % for coding
%\usepackage[french]{babel} % cesures, spaces etc
%% Misc 
\usepackage{xparse}   % \NewDocumentCommand
\usepackage{pifont}
\usepackage{marvosym}
%\usepackage{fp}       % floating point calculations: cf macro GanttLine
%\usepackage{eurosym}  % \euro symbol
%\usepackage{bibtopic}

\usepackage[usenames,dvipsnames]{xcolor}
\usepackage{listings}


\usepackage{fullpage}

%% macros
%%i%%%%%%%%%%%%%%%%%%%%%%%%%%%%%%%%%%%%%%%%%%%%%%%%%%%%%%%%%%%%%%%%%%%%%%%%%%%%%%%
\newif\ifSLIDES
\SLIDESfalse

\input{.latex_location_macros.sty}

%% General macros 
\input{\wmysty/macros-comments.sty}
\input{\wmysty/macros-envs.sty}
\input{\wmysty/macros-math-letters-fontified.sty}
\input{\wmysty/macros-symbols.sty}
\input{\wmysty/macros-algorithms-code.sty}

%% Macros specific to one broad topic/theme 
\input{\wmysty/macros-SBL.sty}
\input{\wmysty/macros-wp-delaunay-voronoi.sty}
\input{\wmysty/macros-wp-space-filling-models.sty}
\input{\wmysty/macros-wp-conformational-analysis.sty}

%% Macros specific to one/a few papers (PhD project)
%\renewcommand{\tored}{\color{black}}
%\renewcommand{\toblue}{\color{black}}

\title{TP}
\author{Jules}

\begin{document}
\maketitle

%=============================================

\section*{Exo 1}
\begin{enumerate}
    \item \texttt{14}
    \item \texttt{34}
    \item L'adresse de la case mémoire à côté de celle de p.
    \item L'adresse de la case du tableau contenant 23 (la case 1)
    \item L'adresse de la case 3 du tableau (celle qui contient 45)
    \item L'adresse 7 de la mémoire.
    \item L'adresse de la case 2 du tableau (celle contenant 34)
    \item L'adresse de la case 1 du tableau (celle contenant 23)
\end{enumerate}

\section*{Exo 2}
\begin{lstlisting}
char *ptab;
char tab[32];
char ch1[] = "Bonjour";
char ch2[15];

strcpy(tab,"QW");
strcpy(ch2,ch1);
ptab = (char *) calloc(10,sizeof(char));
if (~ptab){
    perror("memory problem");
}
strcpy(ptab,"ASDFGHJKL");

printf("tab : %s\tptab : %s\n",tab,ptab);
printf("tab : %c\tptab : %c\n",*tab,*ptab);
printf("tab : %c\tptab : %c\n",tab[1],ptab[1]);
printf("tab : %c\tptab : %c\n",*(tab+1),*(ptab+1));
printf("tab : %c\tptab : %c\n",*tab+1,*ptab+1);
\end{lstlisting}

It returns

\begin{lstlisting}
tab : QW        ptab : ASDFGHJKL
tab : Q ptab : A
tab : W ptab : S
tab : W ptab : S
tab : R ptab : B
\end{lstlisting}

\section*{Exo 3}

\lstinputlisting{tp7-3.c}

\section*{Exo 4}

\begin{lstlisting}
10 20 30 40
30 21
\end{lstlisting}

\end{document}